\documentclass[tate,
book,
paper=a6paper, % 用紙サイズ
% landscape, 
onecolumn, % 一段組 
fontsize=9pt, %フォントサイズ
baselineskip=1.618zw, % 行送り
% head_space=15mm, % ヘッダースペース
% foot_space=12mm, % フッタースペース
% fore-edge=12mm, % 小口の余白
% gutter=12mm, % ノドの余白
hanging_punctuation % ぶら下げ組を有効にする
]{jlreq}

\usepackage{../flexchoon}
\usepackage{../flexwave}
\usepackage{mystyle}
\usepackage{luatexja-fontspec} % フォント設定

\setmainjfont{GenEi Koburi Mincho v6}

\newjfontfamily{\fontKoburi}{GenEi Koburi Mincho v6}
\newjfontfamily{\fontChikugo}{GenEi Chikugo Mincho v3}
\newjfontfamily{\fontShippori}{Shippori Mincho OTF}
\newjfontfamily{\fontIPA}{IPAexGothic}

\begin{document}

\section{flexchoon パッケージ}

\LaTeX で長い長音「ー」を出力するためのパッケージです。

\vspace{0.5\baselineskip}

「うおぉーーーっ!」(長音の連続)

「うおぉ―――っ!」(全角ダッシュ)

↓ これがこうできる。

「うおぉ\flexchoon{3}っ」

{\footnotesize ※この文書のフォントは源暎こぶり明朝}

\vspace{0.5\baselineskip}

\subsection{長さは自由}

長音の長さは自由に設定できます。

\begin{itemize}
    \item 「うおぉ\flexchoon{1}っ!」(長音一文字分)
    \item 「うおぉ\flexchoon{2}っ!」(長音二文字分)
    \item 「うおぉ\flexchoon{3}っ!」(長音三文字分)
    \item 「うおぉ\flexchoon{5}っ!」(長音五文字分)
    \item 「うおぉ\flexchoon{8}っ!」(長音八文字分)
\end{itemize}

\subsection{各種フォントで使用可能}

{\fontKoburi「うおぉ\flexchoon{3}っ!」 源暎こぶり明朝}

{\fontChikugo「うおぉ\flexchoon{3}っ!」源暎ちくご明朝}

{\fontShippori「うおぉ\flexchoon{3}っ!」しっぽり明朝}

{\fontIPA「うおぉ\flexchoon{3}っ!」IPAゴシック}

\section{flexwave パッケージ}

\LaTeX で波形を出力するためのパッケージです。

「うわ〜〜〜っ!」

「うわ〰〰〰っ!」

↓ これがこうできる。

「うわ\flexwave{3}っ」

\end{document}
